\documentclass[journal,12pt,twocolumn]{IEEEtran}

\usepackage{setspace}
\usepackage{gensymb}
\singlespacing
\usepackage[cmex10]{amsmath}

\usepackage{amsthm}

\usepackage{mathrsfs}
\usepackage{txfonts}
\usepackage{stfloats}
\usepackage{bm}
\usepackage{cite}
\usepackage{cases}
\usepackage{subfig}

\usepackage{longtable}
\usepackage{multirow}

\usepackage{enumitem}
\usepackage{mathtools}
\usepackage{steinmetz}
\usepackage{tikz}
\usepackage{circuitikz}
\usepackage{verbatim}
\usepackage{tfrupee}
\usepackage[breaklinks=true]{hyperref}
\usepackage{graphicx}
\usepackage{tkz-euclide}

\usetikzlibrary{calc,math}
\usepackage{listings}
    \usepackage{color}                                            %%
    \usepackage{array}                                            %%
    \usepackage{longtable}                                        %%
    \usepackage{calc}                                             %%
    \usepackage{multirow}                                         %%
    \usepackage{hhline}                                           %%
    \usepackage{ifthen}                                           %%
    \usepackage{lscape}     
\usepackage{multicol}
\usepackage{chngcntr}

\DeclareMathOperator*{\Res}{Res}

\renewcommand\thesection{\arabic{section}}
\renewcommand\thesubsection{\thesection.\arabic{subsection}}
\renewcommand\thesubsubsection{\thesubsection.\arabic{subsubsection}}

\renewcommand\thesectiondis{\arabic{section}}
\renewcommand\thesubsectiondis{\thesectiondis.\arabic{subsection}}
\renewcommand\thesubsubsectiondis{\thesubsectiondis.\arabic{subsubsection}}


\hyphenation{op-tical net-works semi-conduc-tor}
\def\inputGnumericTable{}                                 %%

\newcommand{\permcomb}[4][0mu]{{{}^{#3}\mkern#1#2_{#4}}}
\newcommand{\perm}[1][-3mu]{\permcomb[#1]{P}}
\newcommand*{\comb}[1][-1mu]{\permcomb[#1]{C}}

\lstset{
%language=C,
frame=single, 
breaklines=true,
columns=fullflexible
}
\newcommand{\blank}[1]{\hfil\penalty1000\hfilneg\rule[-3pt]{#1}{0.4pt}} % nice blank underscores

\begin{document}

\newtheorem{theorem}{Theorem}[section]
\newtheorem{problem}{Problem}
\newtheorem{proposition}{Proposition}[section]
\newtheorem{lemma}{Lemma}[section]
\newtheorem{corollary}[theorem]{Corollary}
\newtheorem{example}{Example}[section]
\newtheorem{definition}[problem]{Definition}


\newcommand{\BEQA}{\begin{eqnarray}}
\newcommand{\EEQA}{\end{eqnarray}}
\newcommand{\define}{\stackrel{\triangle}{=}}
\bibliographystyle{IEEEtran}
\raggedbottom
\setlength{\parindent}{0pt}
\providecommand{\mbf}{\mathbf}
\providecommand{\pr}[1]{\ensuremath{\Pr\left(#1\right)}}
\providecommand{\qfunc}[1]{\ensuremath{Q\left(#1\right)}}
\providecommand{\sbrak}[1]{\ensuremath{{}\left[#1\right]}}
\providecommand{\lsbrak}[1]{\ensuremath{{}\left[#1\right.}}
\providecommand{\rsbrak}[1]{\ensuremath{{}\left.#1\right]}}
\providecommand{\brak}[1]{\ensuremath{\left(#1\right)}}
\providecommand{\lbrak}[1]{\ensuremath{\left(#1\right.}}
\providecommand{\rbrak}[1]{\ensuremath{\left.#1\right)}}
\providecommand{\cbrak}[1]{\ensuremath{\left\{#1\right\}}}
\providecommand{\lcbrak}[1]{\ensuremath{\left\{#1\right.}}
\providecommand{\rcbrak}[1]{\ensuremath{\left.#1\right\}}}
\theoremstyle{remark}
\newtheorem{rem}{Remark}
\newcommand{\sgn}{\mathop{\mathrm{sgn}}}
\providecommand{\abs}[1]{\vert#1\vert}
\providecommand{\res}[1]{\Res\displaylimits_{#1}} 
\providecommand{\norm}[1]{\lVert#1\rVert}
%\providecommand{\norm}[1]{\lVert#1\rVert}
\providecommand{\mtx}[1]{\mathbf{#1}}
\providecommand{\mean}[1]{E[ #1 ]}
\providecommand{\fourier}{\overset{\mathcal{F}}{ \rightleftharpoons}}
%\providecommand{\hilbert}{\overset{\mathcal{H}}{ \rightleftharpoons}}
\providecommand{\system}{\overset{\mathcal{H}}{ \longleftrightarrow}}
	%\newcommand{\solution}[2]{\textbf{Solution:}{#1}}
\newcommand{\solution}{\noindent \textbf{Solution: }}
\newcommand{\cosec}{\,\text{cosec}\,}
\providecommand{\dec}[2]{\ensuremath{\overset{#1}{\underset{#2}{\gtrless}}}}
\newcommand{\myvec}[1]{\ensuremath{\begin{pmatrix}#1\end{pmatrix}}}
\newcommand{\mydet}[1]{\ensuremath{\begin{vmatrix}#1\end{vmatrix}}}
\numberwithin{equation}{subsection}
\makeatletter
\@addtoreset{figure}{problem}
\makeatother
\let\StandardTheFigure\thefigure
\let\vec\mathbf
\renewcommand{\thefigure}{\theproblem}
\def\putbox#1#2#3{\makebox[0in][l]{\makebox[#1][l]{}\raisebox{\baselineskip}[0in][0in]{\raisebox{#2}[0in][0in]{#3}}}}
     \def\rightbox#1{\makebox[0in][r]{#1}}
     \def\centbox#1{\makebox[0in]{#1}}
     \def\topbox#1{\raisebox{-\baselineskip}[0in][0in]{#1}}
     \def\midbox#1{\raisebox{-0.5\baselineskip}[0in][0in]{#1}}
\vspace{3cm}
\title{AI1103-Challenging problem }
\author{Name : Aayush Patel, Roll No.: CS20BTECH11001}
\maketitle
\newpage
\bigskip
\renewcommand{\thefigure}{\theenumi}
\renewcommand{\thetable}{\theenumi}
%
\section*{Question}
 Prove by properties of Q-function the following inequality,
 $$1-\exp(-2 \pi) \geq \brak{1-2Q\brak{\dfrac{1}{2}}}^2$$
\section*{Solution}
\textbf{Some Properties of Q function}:
\begin{lemma}\label{even}
    $Q(x)+Q(-x)=1$
\end{lemma}
\begin{lemma}\label{bound}
Chernoff Lower Bound Property
    $$Q(x)\geq f(x)$$
    where $f(x) = \alpha \exp(-\beta x^2)$ and \\[3pt]$\beta > 1, 0 < \alpha \leq \dfrac{\sqrt{2e}\sqrt{\beta-1}}{\sqrt{\pi}\beta} $
\end{lemma}

Simplifying the inequality given in question,
\begin{align}
    1-\exp(-2 \pi)  &\geq \brak{1-2Q\brak{\dfrac{1}{2}}}^2 \\[1.5pt]
    1-\exp(-2 \pi) &\geq 1+4Q^2\brak{\dfrac{1}{2}}^2-4Q\brak{\dfrac{1}{2}}\\[1.5pt]
    -\exp(-2 \pi) &\geq 4Q\brak{\dfrac{1}{2}}\brak{Q\brak{\dfrac{1}{2}}-1} \\[1.5pt]
    \exp(-2 \pi) &\leq 4Q\brak{\dfrac{1}{2}}\brak{1-Q\brak{\dfrac{1}{2}}} \\[1.5pt]
    4Q\brak{\dfrac{1}{2}}\brak{1-Q\brak{\dfrac{1}{2}}}&\geq \exp(-2 \pi)
\end{align}

Using Lemma \ref{even} of Q-functions,
\begin{equation}
    Q(x)+Q(-x)=1
\end{equation}
\begin{align}
    4Q\brak{\dfrac{1}{2}}Q\brak{-\dfrac{1}{2}}\geq \exp(-2 \pi)
\end{align}
\\
% Let $erfc(x)$ denote the Complementary Error function.\\
\begin{lemma}\label{3}
$4Q\brak{\dfrac{1}{2}}Q\brak{-\dfrac{1}{2}}\geq \exp(-2 \pi)$
\end{lemma}
\begin{proof}
Using Lemma \ref{bound} of Q-functions and the fact that $f(x)$ is even.
\begin{align}
    4Q(x\sqrt{2})Q(-x\sqrt{2}) \geq \alpha^{2}\exp(-2\beta x^{2})
\end{align}
Putting $x=\dfrac{1}{2 \sqrt{2}}$, we get,\\
\begin{align}
    4Q\brak{\dfrac{1}{2}}Q\brak{-\dfrac{1}{2}}\geq \alpha^{2}\exp\brak{\dfrac{-\beta}{4}}
\end{align}
For $\beta = 8\pi$, 
\begin{align}
    0 < \alpha \leq \dfrac{\sqrt{2e}\sqrt{8 \pi-1}}{\sqrt{\pi} 8\pi}\\
\end{align} 
 Clearly, denominator is greater than numerator, therefore, $\alpha < 1$
\begin{align}
    4Q\brak{\dfrac{1}{2}}Q\brak{-\dfrac{1}{2}}\geq \alpha^{2}\exp(-2\pi)
\end{align}
where $\alpha < 1$
\\\\
Therefore,
\begin{align}
    4Q\brak{\dfrac{1}{2}}Q\brak{-\dfrac{1}{2}}\geq \exp(-2\pi)
\end{align}
\end{proof}
Hence proved
\end{document}
