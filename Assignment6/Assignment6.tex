\documentclass[journal,12pt,twocolumn]{IEEEtran}

\usepackage{setspace}
\usepackage{gensymb}
\singlespacing
\usepackage[cmex10]{amsmath}

\usepackage{amsthm}

\usepackage{mathrsfs}
\usepackage{txfonts}
\usepackage{stfloats}
\usepackage{bm}
\usepackage{cite}
\usepackage{cases}
\usepackage{subfig}

\usepackage{longtable}
\usepackage{multirow}

\usepackage{enumitem}
\usepackage{mathtools}
\usepackage{steinmetz}
\usepackage{tikz}
\usepackage{circuitikz}
\usepackage{verbatim}
\usepackage{tfrupee}
\usepackage[breaklinks=true]{hyperref}
\usepackage{graphicx}
\usepackage{tkz-euclide}

\usetikzlibrary{calc,math}
\usepackage{listings}
    \usepackage{color}                                            %%
    \usepackage{array}                                            %%
    \usepackage{longtable}                                        %%
    \usepackage{calc}                                             %%
    \usepackage{multirow}                                         %%
    \usepackage{hhline}                                           %%
    \usepackage{ifthen}                                           %%
    \usepackage{lscape}     
\usepackage{multicol}
\usepackage{chngcntr}

\DeclareMathOperator*{\Res}{Res}

\renewcommand\thesection{\arabic{section}}
\renewcommand\thesubsection{\thesection.\arabic{subsection}}
\renewcommand\thesubsubsection{\thesubsection.\arabic{subsubsection}}

\renewcommand\thesectiondis{\arabic{section}}
\renewcommand\thesubsectiondis{\thesectiondis.\arabic{subsection}}
\renewcommand\thesubsubsectiondis{\thesubsectiondis.\arabic{subsubsection}}


\hyphenation{op-tical net-works semi-conduc-tor}
\def\inputGnumericTable{}                                 %%

\newcommand{\permcomb}[4][0mu]{{{}^{#3}\mkern#1#2_{#4}}}
\newcommand{\perm}[1][-3mu]{\permcomb[#1]{P}}
\newcommand*{\comb}[1][-1mu]{\permcomb[#1]{C}}

\lstset{
%language=C,
frame=single, 
breaklines=true,
columns=fullflexible
}
\begin{document}

\newcommand{\BEQA}{\begin{eqnarray}}
\newcommand{\EEQA}{\end{eqnarray}}
\newcommand{\define}{\stackrel{\triangle}{=}}
\bibliographystyle{IEEEtran}
\raggedbottom
\setlength{\parindent}{0pt}
\providecommand{\mbf}{\mathbf}
\providecommand{\pr}[1]{\ensuremath{\Pr\left(#1\right)}}
\providecommand{\qfunc}[1]{\ensuremath{Q\left(#1\right)}}
\providecommand{\sbrak}[1]{\ensuremath{{}\left[#1\right]}}
\providecommand{\lsbrak}[1]{\ensuremath{{}\left[#1\right.}}
\providecommand{\rsbrak}[1]{\ensuremath{{}\left.#1\right]}}
\providecommand{\brak}[1]{\ensuremath{\left(#1\right)}}
\providecommand{\lbrak}[1]{\ensuremath{\left(#1\right.}}
\providecommand{\rbrak}[1]{\ensuremath{\left.#1\right)}}
\providecommand{\cbrak}[1]{\ensuremath{\left\{#1\right\}}}
\providecommand{\lcbrak}[1]{\ensuremath{\left\{#1\right.}}
\providecommand{\rcbrak}[1]{\ensuremath{\left.#1\right\}}}
\theoremstyle{remark}
\newtheorem{rem}{Remark}
\newcommand{\sgn}{\mathop{\mathrm{sgn}}}
\providecommand{\abs}[1]{\vert#1\vert}
\providecommand{\res}[1]{\Res\displaylimits_{#1}} 
\providecommand{\norm}[1]{\lVert#1\rVert}
%\providecommand{\norm}[1]{\lVert#1\rVert}
\providecommand{\mtx}[1]{\mathbf{#1}}
\providecommand{\mean}[1]{E[ #1 ]}
\providecommand{\fourier}{\overset{\mathcal{F}}{ \rightleftharpoons}}
%\providecommand{\hilbert}{\overset{\mathcal{H}}{ \rightleftharpoons}}
\providecommand{\system}{\overset{\mathcal{H}}{ \longleftrightarrow}}
	%\newcommand{\solution}[2]{\textbf{Solution:}{#1}}
\newcommand{\solution}{\noindent \textbf{Solution: }}
\newcommand{\cosec}{\,\text{cosec}\,}
\providecommand{\dec}[2]{\ensuremath{\overset{#1}{\underset{#2}{\gtrless}}}}
\newcommand{\myvec}[1]{\ensuremath{\begin{pmatrix}#1\end{pmatrix}}}
\newcommand{\mydet}[1]{\ensuremath{\begin{vmatrix}#1\end{vmatrix}}}
\numberwithin{equation}{subsection}
\makeatletter
\@addtoreset{figure}{problem}
\makeatother
\let\StandardTheFigure\thefigure
\let\vec\mathbf
\renewcommand{\thefigure}{\theproblem}
\def\putbox#1#2#3{\makebox[0in][l]{\makebox[#1][l]{}\raisebox{\baselineskip}[0in][0in]{\raisebox{#2}[0in][0in]{#3}}}}
     \def\rightbox#1{\makebox[0in][r]{#1}}
     \def\centbox#1{\makebox[0in]{#1}}
     \def\topbox#1{\raisebox{-\baselineskip}[0in][0in]{#1}}
     \def\midbox#1{\raisebox{-0.5\baselineskip}[0in][0in]{#1}}
\vspace{3cm}
\title{AI1103-Assignment 6}
\author{Name : Aayush Patel, Roll No.: CS20BTECH11001}
\maketitle
\newpage
\bigskip
\renewcommand{\thefigure}{\theenumi}
\renewcommand{\thetable}{\theenumi}
%
Latex codes : 
%
\begin{lstlisting}
https://github.com/Aayush-2492/Assignments/tree/main/Assignment6
\end{lstlisting}
\section*{UGC/MATH 2018(June math set-a), Q.104}
Let X and Y be two random variables satisfying $X\geq 0, Y \geq 0, E(X)=3, Var(X)=9, E(Y)=2$ and $Var(Y)=4$. Which of the following statements are correct?
\begin{enumerate}[label=\Alph*)]
    \item $0\leq Cov(X,Y)\leq 4$
    \item $E(XY)\leq 3$
    \item $Var(X+Y)\leq 25$
    \item $E(X+Y)^2\geq 25$
\end{enumerate}
\section*{Solution}
\begin{equation}
    E(X^2) = Var(X)+(E(X))^2 = 18
\end{equation}
Similarly,
\begin{equation}
    E(Y^2) = Var(Y)+(E(Y))^2 = 8
\end{equation}
We can use the Covariance inequality for this question,
\begin{equation}
  ({Cov(X,Y)})^2\leq Var(X)Var(Y)
\end{equation}
The proof of this inequality is as shown,
\begin{align}
\nonumber    Var\brak{\dfrac{X}{\sigma_X}\pm\dfrac{Y}{\sigma_Y}} &= Var\brak{\dfrac{X}{\sigma_X}}+Var\brak{\dfrac{\pm Y}{\sigma_Y}}\\[6pt]
    &+2Cov\brak{\dfrac{X}{\sigma_X},\dfrac{\pm Y}{\sigma_Y}}\\
\nonumber    &=\dfrac{1}{{\sigma_X}^2}Var(X)+\dfrac{1}{{\sigma_Y}^2}Var(Y)\\[6pt]
    &+2 Cov\brak{\dfrac{X}{\sigma_X},\dfrac{\pm Y}{\sigma_Y}}\\[6pt]
    &=2\pm2\dfrac{Cov(X,Y)}{{\sigma_X}{\sigma_Y}}
\end{align}
Since Variance is always positive,
\begin{align}
    Var\brak{\dfrac{X}{\sigma_X}\pm\dfrac{Y}{\sigma_Y}}\geq 0
    \\[6pt]
    2\pm2\dfrac{Cov(X,Y)}{{\sigma_X}{\sigma_Y}} \geq 0
    \\[6pt]
    1\pm1\dfrac{Cov(X,Y)}{{\sigma_X}{\sigma_Y}}\geq 0
    \\[6pt]
    \abs{({Cov(X,Y)})}\leq (\sigma_X)(\sigma_Y)
    \\
    ({Cov(X,Y)})^2\leq Var(X)Var(Y)
\end{align}
\begin{enumerate}
\item
Substituting values of variance we get,
\begin{equation}\label{cov}
    -6\leq{Cov(X,Y)}\leq 6
\end{equation}
\textbf{Therefore, option A is incorrect}.\\
\item
From equation \eqref{cov},
\begin{equation}
    {Cov(X,Y)}=E(XY)-E(X)E(Y)
\end{equation}
\begin{align}
    -6\leq E(XY)-E(X)E(Y)\leq 6
\end{align}
\begin{equation}\label{exy}
    0\leq E(XY)\leq12
\end{equation}
Also, if X and Y are independent,
\begin{equation}
    E(XY)=E(X)E(Y)=6
\end{equation}
\textbf{Therefore, Option B is incorrect.}\\
\item
Now,
\begin{align}
    Var(X+Y)&=Var(X)+Var(Y)+2Cov(X,Y)\\
    &=13+2Cov(X,Y)
\end{align}
From equation \eqref{cov},
\begin{equation}
    1\leq Var(X+Y)\leq 25
\end{equation}
\textbf{Therefore, Option C is correct.}\\
\item
Now,
\begin{align}
    E(X+Y)^2 &= E(X^2)+E(Y^2)+2E(XY)\\
    E(X+Y)^2 &= 26+2E(XY)
\end{align}
From equation \eqref{exy},
\begin{equation}
    26\leq E(X+Y)^2\leq50
\end{equation}
\textbf{Therefore, Option D is correct.}\\
\end{enumerate}
\end{document}
